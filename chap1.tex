\documentclass[11pt]{article}
\usepackage[utf8]{inputenc}
\usepackage[left=1.0in, top=1.0in, bottom=1.0in, right=1.0in]{geometry}
\usepackage[hidelinks]{hyperref}
\usepackage{amssymb}
\usepackage{graphicx}
\usepackage{bbm, bm}                     % blackboard and bold fonts
\usepackage{mathtools}                   % provides starred version of bmatrix. loads amsmath
\usepackage{float}                       % more exact placement of figures
\usepackage{array}                       % justification of fixed-width columns & matrices
\usepackage{arydshln}                    % dashed lines in matrices
\usepackage{titlesec}                    % change title formats
\usepackage[most, breakable]{tcolorbox}  % nicely colored boxes (most allows for no frames)
\usepackage[shortlabels]{enumitem}       % change enumerate symbols
\usepackage{nicefrac}                    % small in-line fractions
\usepackage{booktabs}                    % nice table properties, e.g. toprule & bottomrule
\usepackage{wasysym}                     % larger set of glyphs for, e.g. binary relations
\usepackage[bottom]{footmisc}            % put footnote in environment at bottom of page
\usepackage{threeparttable}              % notes at the table
% \usepackage[capposition=top]{floatrow}   % `floatfoot` command for figure notes

\usepackage{tikz}       % drawing stuff
\usetikzlibrary{matrix} % drawing matrices

\usepackage{stackengine, scalerel} % adding some padding in fractions

\usepackage{siunitx}
\sisetup{
    per-mode = symbol,
    output-decimal-marker = {.},
    % group-minimum-digits = 4,
    input-symbols = (),
    range-units = brackets,
    list-final-separator = { \translate{and} },
    list-pair-separator = { \translate{and} },
    range-phrase = { \translate{to (numerical range)} },
}

%-- enumerate options
\setlist{
    listparindent=\parindent,
    parsep=0pt
}


% trinidad colors
\definecolor{tgreen}{RGB}{57, 115, 82}
\definecolor{tlightgreen}{RGB}{121, 217, 163}
\definecolor{tmarigold}{RGB}{242, 159, 5}
\definecolor{torange}{RGB}{217, 103, 4}
\definecolor{tred}{RGB}{115, 32, 2}
\definecolor{tcoral}{RGB}{242, 121, 131}
\definecolor{tblue}{RGB}{12, 99, 102}
\definecolor{tlightblue}{RGB}{20, 161, 166}
\definecolor{tlava}{RGB}{73, 65, 55}
\definecolor{tcoffee}{RGB}{58, 48, 45}



\titleformat{\section}
    {\large\scshape}{\thesection}{1em}{}
\titleformat{\subsection}
    {\large\scshape}{\thesubsection}{1em}{}

% allows equation and align environments to break over pages
\allowdisplaybreaks

% faster way of writing a norm
\newcommand\norm[1]{\left\lVert#1\right\rVert}

%vectors with angle, parenthesis, & bracket delimiters
\newcommand\avec[1]{\left\langle#1\right\rangle}
\newcommand\pvec[1]{\left(#1\right)}
\newcommand\bvec[1]{\left[#1\right]}

% argmin and argmax operators
\DeclareMathOperator*{\argmax}{argmax}
\DeclareMathOperator*{\argmin}{argmin}

% shortening varepsilon
\newcommand\ve{\varepsilon}

% shortening common bold-faced letters and variance
\newcommand\E{\mathbb{E}}
\newcommand\R{\mathbb{R}}
\newcommand\Var{\text{Var}}
\newcommand\Cov{\text{Cov}}
\renewcommand\P{\mathbb{P}} % \P is the paragraph symbol by default

\newcommand\cF{\mathcal{F}}

% succsim & precsim with slashes through it
%   https://tex.stackexchange.com/a/479194
\makeatletter
\newcommand\nsuccsim{\mathrel{\mathpalette\varn@t\succsim}}
\newcommand\nprecsim{\mathrel{\mathpalette\varn@t\precsim}}
\newcommand\varn@t[2]{%
    \vphantom{/{#2}}%
    \ooalign{\hfil$\m@th#1/\mkern2mu$\cr\hfil$\m@th#1#2$\hfil\cr}%
}
\makeatother

% partial derivatives
\newcommand\dd[2]{\frac{\partial#1}{\partial#2}}
\newcommand\ndd[2]{\nicefrac{\partial#1}{\partial#2}}

\stackMath
\newcommand\tdd[3][1pt]{\tfrac{%
    \ThisStyle{\addstackgap[#1]{\SavedStyle\partial#2}}}{%
    \ThisStyle{\addstackgap[#1]{\SavedStyle\partial#3}}%
}}

% spaced fraction
\stackMath
\newcommand\sfrac[3][1pt]{\tfrac{%
    \ThisStyle{\addstackgap[#1]{\SavedStyle#2}}}{%
    \ThisStyle{\addstackgap[#1]{\SavedStyle#3}}%
}}

% decreases space between tilde and character its above
\newcommand\stilde[1]{\smash{\tilde{#1}}}

% label one equation in align
\newcommand\numberthis{\addtocounter{equation}{1}\tag{\theequation}}

% making smaller bullets in itemize environments
\newlength{\mylen}
\setbox1=\hbox{$\bullet$}\setbox2=\hbox{\tiny$\bullet$}
\setlength{\mylen}{\dimexpr0.5\ht1-0.5\ht2}

% fixed-width, aligned column types
\newcolumntype{L}[1]{>{\raggedright\let\newline\\\arraybackslash\hspace{0pt}}p{#1}}
\newcolumntype{C}[1]{>{\centering\let\newline\\\arraybackslash\hspace{0pt}}p{#1}}
\newcolumntype{R}[1]{>{\raggedleft\let\newline\\\arraybackslash\hspace{0pt}}p{#1}}

% environments for questions and answers
\newcounter{question}[section]
\NewDocumentEnvironment{hwquestion}{o}
    {
        \refstepcounter{question}
        \begin{tcolorbox}[
            colback=tlightgreen,
            colframe=tgreen,
            opacityback=0.25,
            title=\IfNoValueTF{#1}
                {Question~\thequestion}
                {Question~\thequestion \, - #1},
            breakable,
            enhanced jigsaw,
            before upper={\parindent15pt} %-- add paragraph indents
        ]
    }
    {
        \end{tcolorbox}
    }
\newcounter{subquestion}[question]
\NewDocumentEnvironment{hwsubquestion}{o}
    {
        \refstepcounter{subquestion}
        \begin{tcolorbox}[
            colback=tlightgreen,
            colframe=tgreen,
            opacityback=0.25,
            title=\IfNoValueTF{#1}
                {Question~\thequestion.\thesubquestion}
                {Question~\thequestion.\thesubquestion \, - #1},
            breakable,
            enhanced jigsaw,
            before upper={\parindent15pt} %-- add paragraph indents
        ]
    }
    {
        \end{tcolorbox}
    }
\newenvironment{hwanswer}
    {
        \vspace{2mm}
        {\bfseries Answer}
        \vspace{-\abovedisplayskip}
        \begin{center}
            \begin{tcolorbox}[
                width=0.95\textwidth,
                colback=white,
                colframe=white,
                opacityback=0,
                opacityframe=0,
                boxrule=0pt,
                frame hidden,
                breakable,
                before upper={\parindent15pt} %-- add paragraph indents
            ]
            \lineskip=0pt % smashes inline math to same line spacing
    }
    {
        \end{tcolorbox}
        \end{center}
        \vspace{4mm}
    }
\newenvironment{hwquestiongroup}[1][\unskip]
    {
        \begin{center}
            \begin{tcolorbox}[
                width=0.90\linewidth,
                colback=torange,
                colframe=torange,
                opacityback=0.25,
                title= #1,
                breakable,
                enhanced jigsaw
            ]
    }
    {
            \end{tcolorbox}
        \end{center}
    }
\newenvironment{hwlemma}[1][\unskip]
    {
        \vspace{1mm}
        \begin{center}
            \begin{tcolorbox}[
                width=0.95\linewidth,
                colback=tmarigold,
                colframe=tmarigold,
                opacityback=0.10,
                title= #1,
                breakable,
                enhanced jigsaw
            ]
    }
    {
            \end{tcolorbox}
        \end{center}
    }
\newenvironment{hwdefinition}[1][\unskip]
    {
        \vspace{1mm}
        \begin{center}
            \begin{tcolorbox}[
                width=0.95\linewidth,
                colback=tcoral,
                colframe=tcoral,
                opacityback=0.10,
                title= #1,
                breakable,
                enhanced jigsaw
            ]
    }
    {
            \end{tcolorbox}
        \end{center}
    }

% environment for header that includes question numbers
\newenvironment{hwheader}
    {
        \begin{flushright}
            \begin{tcolorbox}[
                width=0.55\textwidth,
                colback=tlightblue,
                colframe=tblue,
                opacityback=0.25,
                enhanced jigsaw
            ]
                \begin{flushright}
                    Logan Hotz \\
    }
    {
                \end{flushright}
            \end{tcolorbox}
        \end{flushright}
        \vspace{4mm}
    }





\begin{document}



    \begin{hwheader}
        Stochastic Calculus for Finance II

        Shreve
    \end{hwheader}





    %-- question 1
    \begin{hwquestion}
        Use the properties of Definition 1.1.2 for a probability measure $\P$, show the
        following
        \begin{enumerate}[(i), nolistsep]
            \item If $A \in \cF, B \in \cF$, and $A \subset B$, then $\P(A) \leq \P(B)$.
            \item If $A \in \cF$ and $\{ A_n \}_{n=1}^{\infty}$ is a sequence of sets in
            $\cF$ with $\lim_{n \to \infty} \P(A_n) = 0$ and $A \subset A_n$ for every $n$,
            then $\P(A) = 0$.
        \end{enumerate}
    \end{hwquestion}

    \begin{hwanswer}
        \begin{enumerate}[(i)]
            \item Write $B$ as $B = A \cup (B \setminus A)$, and note that $A$ and $B
            \setminus A$ are disjoint. From part (ii) of Definition 1.1.2, then, we have
            $\P(B) = \P(A) + \P(B \setminus A)$. Because probabilities are weakly greater
            than zero, $\P(B) = \P(A) + \P(B \setminus A) \geq \P(A)$, as desired.

            \item From part (i), $\P(A) \leq \P(A_n)$ for all $n \in \mathbb{N}$. Again
            making use of the fact $\P(A) \geq 0$, we have
            \[
                0
                =
                \lim_{n \to \infty} 0
                \leq
                \lim_{n \to \infty} \P(A)
                \leq
                \lim_{n \to \infty} \P(A_n)
                = 0.
            \]
            Thus, $\lim_{n \to \infty} \P(A) = P(A) = 0$.
        \end{enumerate}
    \end{hwanswer}





    %-- question 2
    \begin{hwquestion}
        The infinite coin-toss space $\Omega_{\infty}$ of Example 1.1.4 is \emph{uncountably
        infinite}. In other words, we cannot list all its elements in a sequence. To see
        that this is impossible, suppose there such a sequential list of all elements of
        $\Omega_{\infty}$:
        \[
            \begin{aligned}
                \omega^{(1)}
                &=
                \omega_{1}^{(1)}
                \omega_{2}^{(1)}
                \omega_{3}^{(1)}
                \omega_{4}^{(1)}
                \, \cdots, \\
                %
                %
                \omega^{(2)}
                &=
                \omega_{1}^{(2)}
                \omega_{2}^{(2)}
                \omega_{3}^{(2)}
                \omega_{4}^{(2)}
                \, \cdots, \\
                %
                %
                \omega^{(3)}
                &=
                \omega_{1}^{(3)}
                \omega_{2}^{(3)}
                \omega_{3}^{(3)}
                \omega_{4}^{(3)}
                \, \cdots, \\
                %
                %
                {}
                &\vdots
                {}
            \end{aligned}
        \]
        An element that does not appear in this list is the sequence whose first component
        is $H$ if $\omega_{1}^{(1)}$ is $T$ and is $T$ if $\omega_{1}^{(1)}$ is $H$, whose
        second component is $H$ if $\omega_{2}^{(2)}$ is $T$ and is $T$ if $\omega_{2}^{
        (2)}$ is $H$, whose third component is $H$ if $\omega_{3}^{(3)}$ is $T$ and is
        $T$ if $\omega_{3}^{(3)}$ is $H$, etc. Thus, the list does not include every element
        of $\Omega_{\infty}$.

        Now consider the set of sequences of coin tosses in which the outcome on each
        even-numbered toss matches the outcome of the toss preceding it, i.e.,
        \[
            A = \big\{
                \omega =
                \omega_1
                \omega_2
                \omega_3
                \omega_4
                \omega_5
                \cdots
                \, | \,
                \omega_1
                =
                \omega_2
                , \, 
                \omega_3
                =
                \omega_4
                \cdots
            \big\}.
        \]

        \vspace{2mm}

        \begin{enumerate}[(i), nolistsep]
            \item Show that $A$ is uncountably infinite.
            \item Show that, when $0 < p < 1$, we have $\P(A) = 0$.
        \end{enumerate}

        \noindent Uncountably infinite sets can have any probability between zero and one,
        including zero and one. The uncountability of a set does not determine its
        probability.
    \end{hwquestion}

    \begin{hwanswer}
        \begin{enumerate}[(i), nolistsep]
            \item We use the same type of diagonalization argument as we did for
            $\Omega_{\infty}$.

            Define the ``negation'' operator for coin flips as $\neg$ as $\neg H = T$ and
            $\neg T = H$ for convenience. Suppose $L = \{ \omega^{(n)} \}$ is a sequential
            list of all the elements of $A$, in the same manner as in the question
            statement. Consider the element $\widetilde{\omega} \in \Omega_{\infty}$ with
            the $n$-th coinflip given by $\widetilde{\omega}_{n} = \neg \omega_{2n-1}^{(n)}$.
            That is, the $n$-th coinflip in $\widetilde{\omega}$ is the negation of the
            $(2n-1)$-th paired coinflip of the $n$-th element of $L$. Clearly, $\widetilde{
            \omega}$ is not in $L$, and thus $A$ is uncountably infinite.

            \item Recall $p$ is the probability of a heads. For $n \in \mathbb{N}$, define
            $
            A_n = \big\{
                \omega =
                \omega_1
                \omega_2
                \cdots
                \omega_{2n-1}
                \omega_{2n}
                \, | \,
                \omega_1
                =
                \omega_2
                \cdots
                \omega_{2n-1}
                =
                \omega_{2n}
                \cdots
            \big\}
            $
            to be the set of $n$ paired coinflips. The sequence of sets $\{ A_n \}$
            converges to $A$ as $n \to \infty$ Thus
            \[
                \P(A)
                =
                \lim_{n \to \infty} \P(A_n)
                =
                \lim_{n \to \infty}
                \P(\omega_1 = \omega_2)
                \times
                \cdots
                \times
                \P(\omega_{2n-1} = \omega_{2n})
                =
                \lim_{n \to \infty}
                \big[ p^2 + (1 - p)^2 \big]^n.
            \]
            For $0 < p < 1$, the term $p^2 + (1 - p)^2$ is strictly less than one. Thus
            $\lim_{n \to \infty} \big[ p^2 + (1 - p)^2 \big]^n$ converges to zero, and
            $\P(A) = 0$.
        \end{enumerate}
    \end{hwanswer}





    %-- question 3
    \begin{hwquestion}
        Consider the set function $\P$ defined for every subset of $[0, 1]$ by the formula
        that $\P(A) = 0$ if $A$ is finite and $\P(A) = \infty$ if $A$ is an infinite set.
        Show that $\P$ satisfies (1.1.3) - (1.1.5), but $\P$ does not have the countable
        additive property (1.1.2). We see then that the finite additivity property (1.1.5)
        does not imply the countable additivity property (1.1.2).
    \end{hwquestion}

    \begin{hwanswer}
        The four properties are:
        \begin{itemize}
            \item (1.1.2): Whenever $A_1, A_2, \dots$ is a sequence of disjoint sets in
            $\cF$, then
            \[
                \P\left(
                    \bigcup_{n=1}^{\infty} A_n
                \right)
                =
                \sum_{n=1}^{\infty} \P(A_n).
            \]
            \item (1.1.3): $\P(\varnothing) = 0$.
            \item (1.1.4): If $A$ and $B$ are disjoint sets in $\cF$, $\P(A \cup B) = \P(A)
            + \P(B)$.
            \item (1.1.5): If $A_1, A_2, \dots, A_N$ are finitely many disjoint sets in $\cF$,
            then
            \[
                \P\left(
                    \bigcup_{n=1}^{N} A_n
                \right)
                =
                \sum_{n=1}^{N} \P(A_n).
            \]
        \end{itemize}

        The empty set is finite, hence $\P(\varnothing) = 0$ is immediate.

        Let $A, B \subseteq [0, 1]$. If $A$ and $B$ are finite, so too is $A \cup B$, so
        that $\P(A \cup B) = 0 = 0 + 0 = \P(A) + \P(B)$. If instead $A$ is finite and $B$ is
        infinite, so is $A \cup B$, meaning $\P(A \cup B) = \infty = 0 + \infty = \P(A) +
        \P(B)$. The case where $A$ is infinite and $B$ is finite is similar. If both $A$ and
        $B$ are infinite, then so too is $A \cup B$. Thus $\P(A \cup B) = \infty = \infty +
        \infty = \P(A) + \P(B)$.

        Let $A_1, \dots A_N$ be a collection of disjoint sets in $[0, 1]$. If at least one
        of the $A_n$ is infinite, then $\cup_{n} A_n$ is as well. Thus
        $%\[
            \P\left(
                \cup_{n=1}^{N} A_n
            \right)
            =
            \infty
            =
            \sum_{n=1}^{N} A_n.
        $%\]
        If instead all the $A_{n}$ are finite, then the finite union $\cup_{n} A_n$ is too.
        Therefore we can iteratively apply (1.1.4) to $\cup_{n} A_n$ to deduce that the
        probability of the union and the sum of the individual probabilities are both zero.

        To show that this set function does not satisfy the countable additivity property,
        let $A_{n} = \nicefrac{1}{n}$ for $n \in \mathbb{N}$. Then $\P(A_{n}) = 0$ for all
        $n$, but $A = \cup_{n} A_{n} = \{ \nicefrac{1}{n} \, | \, n \in \mathbb{N} \}$ is an
        infinite set, and thus we have $\P(A) = \infty$.
    \end{hwanswer}





    %-- question 4
    \begin{hwquestion}
        \begin{enumerate}[(i), nolistsep]
            \item Construct a standard normal random variable $Z$ on the probability space
            $(\Omega_{\infty}, \cF_{\infty}, \P)$ of Example 1.1.4 under the assumption that
            the probability for heads is $p = \nicefrac{1}{2}$.

            \item Define a sequence of random variables $\{ Z_n \}_{n=1}^{\infty}$ on
            $\Omega_{\infty}$ such that
            \[
                \lim_{n \to \infty} Z_n(\omega) = Z(\omega)
                \quad \text{ for every } \quad
                \omega \in \Omega_{\infty}
            \]
            and, for each $n$, $Z_n$ depends only on the first $n$ coin tosses. (This gives
            us a procedure for approximating a standard normal random variable by random
            variables generated by a finite number of coin tosses, a useful algorithm for
            Monte Carlo simulation.)
        \end{enumerate}
    \end{hwquestion}





\end{document}
